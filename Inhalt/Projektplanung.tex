% !TEX root = ../Projektdokumentation.tex
\section{Projektplanung} 
\label{sec:Projektplanung}


\subsection{Projektphasen}
\label{sec:Projektphasen}

\begin{itemize}
	\item In welchem Zeitraum und unter welchen Rahmenbedingungen (\zB Tagesarbeitszeit) findet das Projekt statt?
	\item Verfeinerung der Zeitplanung, die bereits im Projektantrag vorgestellt wurde.
\end{itemize}

\paragraph{Beispiel}
Tabelle~\ref{tab:Zeitplanung} zeigt ein Beispiel für eine grobe Zeitplanung.
\tabelle{Zeitplanung}{tab:Zeitplanung}{ZeitplanungKurz}\\
Eine detailliertere Zeitplanung findet sich im \Anhang{app:Zeitplanung}.


\subsection{Abweichungen vom Projektantrag}
\label{sec:AbweichungenProjektantrag}

\begin{itemize}
	\item Sollte es Abweichungen zum Projektantrag geben (\zB Zeitplanung, Inhalt des Projekts, neue Anforderungen), müssen diese explizit aufgeführt und begründet werden.
\end{itemize}


\subsection{Ressourcenplanung}
\label{sec:Ressourcenplanung}

\begin{itemize}
	\item Detaillierte Planung der benötigten Ressourcen (Hard-/Software, Räumlichkeiten \usw).
	\item \Ggfs sind auch personelle Ressourcen einzuplanen (\zB unterstützende Mitarbeiter).
	\item Hinweis: Häufig werden hier Ressourcen vergessen, die als selbstverständlich angesehen werden (\zB PC, Büro). 
\end{itemize}


\subsection{Entwicklungsprozess}
\label{sec:Entwicklungsprozess}
\begin{itemize}
	\item Welcher Entwicklungsprozess wird bei der Bearbeitung des Projekts verfolgt (\zB Wasserfall, agiler Prozess)?
\end{itemize}
